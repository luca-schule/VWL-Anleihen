\documentclass[10pt,a4paper, ngerman]{beamer}

\include{beamer}

\AtBeginSection{\frame{\frametitle{Gliederung}\tableofcontents[currentsection]}}

\author{Luca Kiebel}
\title{Anleihen}
%\subtitle{subtitle}
\date{\today}
\institute[HBBK]{Hans-Böckler-Berufskolleg}
\setlength{\itemsep}{10pt}
\begin{document}
\begin{frame}
\titlepage
\end{frame}

\section{Einleitung}
\subsection{Was ist eine Anleihe?}
\begin{frame}{Was ist eine Anleihe?}{Einleitung}
\pftn{https://de.wikipedia.org/wiki/Anleihe}
\begin{itemize}
	\item Wertpapier
	\item Gibt Gläubiger Recht auf Rückzahlung + Zinsen
	\item Helfen Schuldner zur Fremdfinanzierung
	\item Helfen Gläubiger zur Kapitalanlage
\end{itemize}
\end{frame}

\subsection{Arten von Anleihen}
\begin{frame}{Arten von Anleihen}{Einleitung}
\pftn{https://de.wikipedia.org/wiki/Anleihe}
\begin{itemize}
	\item Staatsanleihen
	\begin{itemize}
		\item<2-> Staat ist Schuldner
		\item<2-> Bonität bestimmt durch Staatshaushalt
	\end{itemize}
	\item Unternehmensanleihen
	\begin{itemize}
		\item<3-> Unternehmen ist Schuldner
		\item<3-> Bonität bestimmt durch Rating
	\end{itemize}
	\item Pfandbriefe
	\begin{itemize}
		\item<4-> Hypotheken auf Immobilien
		\item<4-> Bonität ist durch Pfand bestimmt
	\end{itemize}
\end{itemize}
\end{frame}

\section{Rendite}
\begin{frame}{Rendite}
\begin{itemize}
	\item 
\end{itemize}
\end{frame}

\section{Liquidität}


\section{Kosten}


\section{Voraussetzungen}
\subsection{Mindesteinlage}
\begin{frame}{Mindesteinlage}{Vorraussetzungen}
\pftn{Judith Engst \& Janne Kipp{,} \textit{Erfolgreiche Geldanlage für Dummies}{,} Wiley, 2017 (Seite 156)}

\end{frame}


\section{Nachhaltigkeit}


\end{document}
