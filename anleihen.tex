\documentclass[10pt,a4paper, ngerman]{beamer}

\include{beamer}
\usepackage{eurosym}

\AtBeginSection{\frame{\frametitle{Gliederung}\tableofcontents[currentsection]}}

\author{Luca Kiebel}
\title{Anleihen}
%\subtitle{subtitle}
\date{\today}
\institute[HBBK]{Hans-Böckler-Berufskolleg}
\setlength{\itemsep}{10pt}
\begin{document}
\begin{frame}
\titlepage
\end{frame}

\section{Einleitung}
\subsection{Was ist eine Anleihe?}
\begin{frame}{Was ist eine Anleihe?}{Einleitung}
\pftn{https://de.wikipedia.org/wiki/Anleihe}
\begin{itemize}
	\item Wertpapier
	\item Gibt Gläubiger Recht auf Rückzahlung + Zinsen
	\item Helfen Schuldner zur Fremdfinanzierung
	\item Helfen Gläubiger zur Kapitalanlage
\end{itemize}
\end{frame}

\subsection{Arten von Anleihen}
\begin{frame}{Arten von Anleihen}{Einleitung}
\pftn{https://de.wikipedia.org/wiki/Anleihe}
\begin{itemize}
	\item Staatsanleihen
	\begin{itemize}
		\item<2-> Staat ist Schuldner
		\item<2-> Bonität bestimmt durch Staatshaushalt
	\end{itemize}
	\item Unternehmensanleihen
	\begin{itemize}
		\item<3-> Unternehmen ist Schuldner
		\item<3-> Bonität bestimmt durch Rating
	\end{itemize}
	\item Pfandbriefe
	\begin{itemize}
		\item<4-> Hypotheken auf Immobilien
		\item<4-> Bonität ist durch Pfand bestimmt
	\end{itemize}
\end{itemize}
\end{frame}

\section{Rendite}
\begin{frame}{Rendite}{Berechnung}
\[Rendite = 100*\left ( \frac{\left ( Zinskoupon + \frac{Verkaufskurs - Kaufkurs}{Laufzeit} \right )}{Kaufkurs} \right )\]
\pftn{https://www.gevestor.de/details/effektivverzinsung-bei-anleihen-so-wird-sie-berechnet-619805.html}
\pause
\begin{block}{Beispiel: Bundesanleihe (10 Jahre)}
\textbf{Zinskoupon}: 0,5 \% \textbf{Kaufkurs}: 65 \% \textbf{Verkaufskurs}: 75 \% \\
--> \textbf{Rendite}: 2,31 \%
\end{block}
\pftn{https://www.deutsche-finanzagentur.de/de/institutionelle-investoren/bundeswertpapiere/bundesanleihen/}
\pftn{https://www.finanzen.net/zinsen/10j-Bundesanleihen}
\end{frame}

\section{Liquidität}
\begin{frame}{Liquidität}
\begin{itemize}
	\item Können jederzeit an der Börse gehandelt werden
	\item Teilweise Handel von Person zu Person möglich \pause
	\item Kosten für Handel an der Börse
\end{itemize}
\pftn{https://www.finanztip.de/anleihen/staatsanleihen/}
\end{frame}

%\section{Kosten}
%\begin{frame}
%
%\end{frame}

\section{Voraussetzungen}
\subsection{Mindesteinlage}
\begin{frame}{Mindesteinlage}{Vorraussetzungen}
\pftn{Judith Engst \& Janne Kipp{,} \textit{Erfolgreiche Geldanlage für Dummies}{,} Wiley, 2017 (Seite 156)}
\begin{itemize}
	\item Deutsche Bundesanleihe: 1 cent\ftn{1}{https://www.deutsche-finanzagentur.de/de/factsheet/sheet-detail/productdata/sheet/DE0001102440/}
	\item Unternehmensanleihen: Meist höher (>20.000 \euro)
\end{itemize}
\end{frame}


\section{Nachhaltigkeit}
\begin{frame}{Nachhatigkeit}
\begin{itemize}
	\item Kommt auf den Schuldner an
	\item Green Bonds: Umweltfreundlich
\end{itemize}
\pftn{https://www.thebalance.com/what-are-green-bonds-417154}
\pftn{Judith Engst \& Janne Kipp{,} \textit{Erfolgreiche Geldanlage für Dummies}{,} Wiley, 2017 (Seite 180)}
\end{frame}

\end{document}
